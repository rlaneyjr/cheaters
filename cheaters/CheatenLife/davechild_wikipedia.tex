\documentclass[10pt,a4paper]{article}

% Packages
\usepackage{fancyhdr}           % For header and footer
\usepackage{multicol}           % Allows multicols in tables
\usepackage{tabularx}           % Intelligent column widths
\usepackage{tabulary}           % Used in header and footer
\usepackage{hhline}             % Border under tables
\usepackage{graphicx}           % For images
\usepackage{xcolor}             % For hex colours
%\usepackage[utf8x]{inputenc}    % For unicode character support
\usepackage[T1]{fontenc}        % Without this we get weird character replacements
\usepackage{colortbl}           % For coloured tables
\usepackage{setspace}           % For line height
\usepackage{lastpage}           % Needed for total page number
\usepackage{seqsplit}           % Splits long words.
%\usepackage{opensans}          % Can't make this work so far. Shame. Would be lovely.
\usepackage[normalem]{ulem}     % For underlining links
% Most of the following are not required for the majority
% of cheat sheets but are needed for some symbol support.
\usepackage{amsmath}            % Symbols
\usepackage{MnSymbol}           % Symbols
\usepackage{wasysym}            % Symbols
%\usepackage[english,german,french,spanish,italian]{babel}              % Languages

% Document Info
\author{Dave Child (DaveChild)}
\pdfinfo{
  /Title (wikipedia.pdf)
  /Creator (Cheatography)
  /Author (Dave Child (DaveChild))
  /Subject (Wikipedia Cheat Sheet)
}

% Lengths and widths
\addtolength{\textwidth}{6cm}
\addtolength{\textheight}{-1cm}
\addtolength{\hoffset}{-3cm}
\addtolength{\voffset}{-2cm}
\setlength{\tabcolsep}{0.2cm} % Space between columns
\setlength{\headsep}{-12pt} % Reduce space between header and content
\setlength{\headheight}{85pt} % If less, LaTeX automatically increases it
\renewcommand{\footrulewidth}{0pt} % Remove footer line
\renewcommand{\headrulewidth}{0pt} % Remove header line
\renewcommand{\seqinsert}{\ifmmode\allowbreak\else\-\fi} % Hyphens in seqsplit
% This two commands together give roughly
% the right line height in the tables
\renewcommand{\arraystretch}{1.3}
\onehalfspacing

% Commands
\newcommand{\SetRowColor}[1]{\noalign{\gdef\RowColorName{#1}}\rowcolor{\RowColorName}} % Shortcut for row colour
\newcommand{\mymulticolumn}[3]{\multicolumn{#1}{>{\columncolor{\RowColorName}}#2}{#3}} % For coloured multi-cols
\newcolumntype{x}[1]{>{\raggedright}p{#1}} % New column types for ragged-right paragraph columns
\newcommand{\tn}{\tabularnewline} % Required as custom column type in use

% Font and Colours
\definecolor{HeadBackground}{HTML}{333333}
\definecolor{FootBackground}{HTML}{666666}
\definecolor{TextColor}{HTML}{333333}
\definecolor{DarkBackground}{HTML}{A3A3A3}
\definecolor{LightBackground}{HTML}{EFEFEF}
\renewcommand{\familydefault}{\sfdefault}
\color{TextColor}

% Header and Footer
\pagestyle{fancy}
\fancyhead{} % Set header to blank
\fancyfoot{} % Set footer to blank
\fancyhead[L]{
\noindent
\begin{multicols}{3}
\begin{tabulary}{5.8cm}{C}
    \SetRowColor{DarkBackground}
    \vspace{-7pt}
    {\parbox{\dimexpr\textwidth-2\fboxsep\relax}{\noindent
        \hspace*{-6pt}\includegraphics[width=5.8cm]{/web/www.cheatography.com/public/images/cheatography_logo.pdf}}
    }
\end{tabulary}
\columnbreak
\begin{tabulary}{11cm}{L}
    \vspace{-2pt}\large{\bf{\textcolor{DarkBackground}{\textrm{Wikipedia Cheat Sheet}}}} \\
    \normalsize{by \textcolor{DarkBackground}{Dave Child (DaveChild)} via \textcolor{DarkBackground}{\uline{cheatography.com/1/cs/155/}}}
\end{tabulary}
\end{multicols}}

\fancyfoot[L]{ \footnotesize
\noindent
\begin{multicols}{3}
\begin{tabulary}{5.8cm}{LL}
  \SetRowColor{FootBackground}
  \mymulticolumn{2}{p{5.377cm}}{\bf\textcolor{white}{Cheatographer}}  \\
  \vspace{-2pt}Dave Child (DaveChild) \\
  \uline{cheatography.com/davechild} \\
        \uline{\seqsplit{www}.getpostcookie.com}
  \end{tabulary}
\vfill
\columnbreak
\begin{tabulary}{5.8cm}{L}
  \SetRowColor{FootBackground}
  \mymulticolumn{1}{p{5.377cm}}{\bf\textcolor{white}{Cheat Sheet}}  \\
   \vspace{-2pt}Published 2nd January, 2012.\\
   Updated 9th February, 2015.\\
   Page {\thepage} of \pageref{LastPage}.
\end{tabulary}
\vfill
\columnbreak
\begin{tabulary}{5.8cm}{L}
  \SetRowColor{FootBackground}
  \mymulticolumn{1}{p{5.377cm}}{\bf\textcolor{white}{Sponsor}}  \\
  \SetRowColor{white}
  \vspace{-5pt}
  %\includegraphics[width=48px,height=48px]{dave.jpeg}
  Measure your website readability!\\
  www.readability-score.com
\end{tabulary}
\end{multicols}}




\begin{document}
\raggedright
\raggedcolumns

% Set font size to small. Switch to any value
% from this page to resize cheat sheet text:
% www.emerson.emory.edu/services/latex/latex_169.html
\footnotesize % Small font.

\begin{multicols*}{3}

\begin{tabularx}{5.377cm}{x{0.84609 cm} x{4.13091 cm} }
\SetRowColor{DarkBackground}
\mymulticolumn{2}{x{5.377cm}}{\bf\textcolor{white}{Five Pillars}}  \tn
% Row 0
\SetRowColor{LightBackground}
\mymulticolumn{2}{x{5.377cm}}{The "Five Pillars" are the fundamental principles of Wikipedia:} \tn 
% Row Count 2 (+ 2)
% Row 1
\SetRowColor{white}
\{\{ar\}\}1. & \{\{link="https://en.wikipedia.org/wiki/Wikipedia:What\_Wikipedia\_is\_not"\}\}Wikipedia is an encyclopedia \tn 
% Row Count 6 (+ 4)
% Row 2
\SetRowColor{LightBackground}
\{\{ar\}\}2. & \{\{link="https://en.wikipedia.org/wiki/Wikipedia:Neutral\_point\_of\_view"\}\}Wikipedia is written from a neutral point of view \tn 
% Row Count 10 (+ 4)
% Row 3
\SetRowColor{white}
\{\{ar\}\}3. & \{\{link="https://en.wikipedia.org/wiki/Wikipedia:Copyrights"\}\}Wikipedia is free content that anyone can use, edit, and distribute \tn 
% Row Count 14 (+ 4)
% Row 4
\SetRowColor{LightBackground}
\{\{ar\}\}4. & \{\{link="https://en.wikipedia.org/wiki/Wikipedia:Civility"\}\}Editors should treat each other with respect and civility \tn 
% Row Count 18 (+ 4)
% Row 5
\SetRowColor{white}
\{\{ar\}\}5. & \{\{link="https://en.wikipedia.org/wiki/Wikipedia:Ignore\_all\_rules"\}\}Wikipedia has no firm rules \tn 
% Row Count 21 (+ 3)
\hhline{>{\arrayrulecolor{DarkBackground}}--}
\end{tabularx}
\par\addvspace{1.3em}

\begin{tabularx}{5.377cm}{x{2.18988 cm} x{2.78712 cm} }
\SetRowColor{DarkBackground}
\mymulticolumn{2}{x{5.377cm}}{\bf\textcolor{white}{Wikipedia Headings}}  \tn
% Row 0
\SetRowColor{LightBackground}
==Text== & Level 1 heading \tn 
% Row Count 1 (+ 1)
% Row 1
\SetRowColor{white}
===Text=== & Level 2 \tn 
% Row Count 2 (+ 1)
% Row 2
\SetRowColor{LightBackground}
====Text==== & Level 3 \tn 
% Row Count 3 (+ 1)
\hhline{>{\arrayrulecolor{DarkBackground}}--}
\end{tabularx}
\par\addvspace{1.3em}

\begin{tabularx}{5.377cm}{x{3.08574 cm} x{1.89126 cm} }
\SetRowColor{DarkBackground}
\mymulticolumn{2}{x{5.377cm}}{\bf\textcolor{white}{Wikipedia Text Effects}}  \tn
% Row 0
\SetRowColor{LightBackground}
''text'' & Italics \tn 
% Row Count 1 (+ 1)
% Row 1
\SetRowColor{white}
'''text''' & Bold \tn 
% Row Count 2 (+ 1)
% Row 2
\SetRowColor{LightBackground}
\textless{}s\textgreater{}text\textless{}/s\textgreater{} & Strikethrough \tn 
% Row Count 3 (+ 1)
% Row 3
\SetRowColor{white}
\textless{}sup\textgreater{}text\textless{}/sup\textgreater{} & Superscript \tn 
% Row Count 4 (+ 1)
% Row 4
\SetRowColor{LightBackground}
\textless{}nowiki\textgreater{}no ''markup''\textless{}/nowiki\textgreater{} & Escape wiki markup \tn 
% Row Count 6 (+ 2)
\hhline{>{\arrayrulecolor{DarkBackground}}--}
\end{tabularx}
\par\addvspace{1.3em}

\begin{tabularx}{5.377cm}{X}
\SetRowColor{DarkBackground}
\mymulticolumn{1}{x{5.377cm}}{\bf\textcolor{white}{Wikipedia Links}}  \tn
% Row 0
\SetRowColor{LightBackground}
\mymulticolumn{1}{x{5.377cm}}{{[}{[}Article Name{]}{]}} \tn 
\mymulticolumn{1}{x{5.377cm}}{\hspace*{6 px}\rule{2px}{6px}\hspace*{6 px}Link to article "Article Name"} \tn 
% Row Count 2 (+ 2)
% Row 1
\SetRowColor{white}
\mymulticolumn{1}{x{5.377cm}}{{[}{[}Article Name{]}{]}s} \tn 
\mymulticolumn{1}{x{5.377cm}}{\hspace*{6 px}\rule{2px}{6px}\hspace*{6 px}Link with text  "Article Names"} \tn 
% Row Count 4 (+ 2)
% Row 2
\SetRowColor{LightBackground}
\mymulticolumn{1}{x{5.377cm}}{{[}{[}Article Name|Link Text{]}{]}} \tn 
\mymulticolumn{1}{x{5.377cm}}{\hspace*{6 px}\rule{2px}{6px}\hspace*{6 px}Link with text "Link Text"} \tn 
% Row Count 6 (+ 2)
% Row 3
\SetRowColor{white}
\mymulticolumn{1}{x{5.377cm}}{{[}{[}Article Name\#Subsection|Link Text{]}{]}} \tn 
\mymulticolumn{1}{x{5.377cm}}{\hspace*{6 px}\rule{2px}{6px}\hspace*{6 px}Link to subsection of article} \tn 
% Row Count 8 (+ 2)
% Row 4
\SetRowColor{LightBackground}
\mymulticolumn{1}{x{5.377cm}}{{[}http://www.example.com Link Text{]}} \tn 
\mymulticolumn{1}{x{5.377cm}}{\hspace*{6 px}\rule{2px}{6px}\hspace*{6 px}Link to example.com} \tn 
% Row Count 10 (+ 2)
% Row 5
\SetRowColor{white}
\mymulticolumn{1}{x{5.377cm}}{{[}{[}Category:Category Name{]}{]}} \tn 
\mymulticolumn{1}{x{5.377cm}}{\hspace*{6 px}\rule{2px}{6px}\hspace*{6 px}Add article to "Category Name"} \tn 
% Row Count 12 (+ 2)
% Row 6
\SetRowColor{LightBackground}
\mymulticolumn{1}{x{5.377cm}}{{[}{[}:Category:Category Name{]}{]}} \tn 
\mymulticolumn{1}{x{5.377cm}}{\hspace*{6 px}\rule{2px}{6px}\hspace*{6 px}Link to category} \tn 
% Row Count 14 (+ 2)
\hhline{>{\arrayrulecolor{DarkBackground}}-}
\end{tabularx}
\par\addvspace{1.3em}

\begin{tabularx}{5.377cm}{x{1.4931 cm} x{3.4839 cm} }
\SetRowColor{DarkBackground}
\mymulticolumn{2}{x{5.377cm}}{\bf\textcolor{white}{Wikipedia Lists and Indents}}  \tn
% Row 0
\SetRowColor{LightBackground}
* Text & Bullet list \tn 
% Row Count 1 (+ 1)
% Row 1
\SetRowColor{white}
** Text & Bullet list (second level) \tn 
% Row Count 2 (+ 1)
% Row 2
\SetRowColor{LightBackground}
\# Text & Numbered list \tn 
% Row Count 3 (+ 1)
% Row 3
\SetRowColor{white}
\#\# Text & Numbered list (second level) \tn 
% Row Count 4 (+ 1)
% Row 4
\SetRowColor{LightBackground}
; Item & Definition list item \tn 
% Row Count 5 (+ 1)
% Row 5
\SetRowColor{white}
: Definition & Definition list definition \tn 
% Row Count 6 (+ 1)
% Row 6
\SetRowColor{LightBackground}
: Text & Indented text \tn 
% Row Count 7 (+ 1)
% Row 7
\SetRowColor{white}
:: Text & Indented text (second level) \tn 
% Row Count 8 (+ 1)
\hhline{>{\arrayrulecolor{DarkBackground}}--}
\SetRowColor{LightBackground}
\mymulticolumn{2}{x{5.377cm}}{Can be nested, e.g., use {\bf{\#*}} to put a bullet list within a numbered list.}  \tn 
\hhline{>{\arrayrulecolor{DarkBackground}}--}
\end{tabularx}
\par\addvspace{1.3em}

\begin{tabularx}{5.377cm}{x{2.04057 cm} x{2.93643 cm} }
\SetRowColor{DarkBackground}
\mymulticolumn{2}{x{5.377cm}}{\bf\textcolor{white}{Wikipedia Lines and Signatures}}  \tn
% Row 0
\SetRowColor{LightBackground}
\{\{literal\}\}-{}-{}-{}- & Horizontal Line \tn 
% Row Count 1 (+ 1)
% Row 1
\SetRowColor{white}
\{\{literal\}\}\textless{}br /\textgreater{} & Line break \tn 
% Row Count 3 (+ 2)
% Row 2
\SetRowColor{LightBackground}
\{\{literal\}\}\textasciitilde{}\textasciitilde{}\textasciitilde{} & Signature \tn 
% Row Count 4 (+ 1)
% Row 3
\SetRowColor{white}
\{\{literal\}\}\textasciitilde{}\textasciitilde{}\textasciitilde{}\textasciitilde{} & Signature with timestamp \tn 
% Row Count 6 (+ 2)
% Row 4
\SetRowColor{LightBackground}
\{\{literal\}\}\textasciitilde{}\textasciitilde{}\textasciitilde{}\textasciitilde{}\textasciitilde{} & Timestamp \tn 
% Row Count 7 (+ 1)
\hhline{>{\arrayrulecolor{DarkBackground}}--}
\end{tabularx}
\par\addvspace{1.3em}

\begin{tabularx}{5.377cm}{x{3.63321 cm} p{1.34379 cm} }
\SetRowColor{DarkBackground}
\mymulticolumn{2}{x{5.377cm}}{\bf\textcolor{white}{Wikipedia Allowed HTML Tags}}  \tn
% Row 0
\SetRowColor{LightBackground}
\textless{}!-{}- Comment -{}-\textgreater{} & \textless{}pre\textgreater{} \tn 
% Row Count 1 (+ 1)
% Row 1
\SetRowColor{white}
\textless{}abbr\textgreater{} & \textless{}tt\textgreater{} \tn 
% Row Count 2 (+ 1)
% Row 2
\SetRowColor{LightBackground}
\textless{}div\textgreater{} & \textless{}font\textgreater{} \tn 
% Row Count 3 (+ 1)
% Row 3
\SetRowColor{white}
\textless{}span\textgreater{} & \textless{}del\textgreater{} \tn 
% Row Count 4 (+ 1)
% Row 4
\SetRowColor{LightBackground}
\textless{}br /\textgreater{} & \textless{}ins\textgreater{} \tn 
% Row Count 5 (+ 1)
% Row 5
\SetRowColor{white}
\textless{}blockquote\textgreater{} & \textless{}sub\textgreater{} \tn 
% Row Count 6 (+ 1)
% Row 6
\SetRowColor{LightBackground}
\textless{}code\textgreater{} & \textless{}sup\textgreater{} \tn 
% Row Count 7 (+ 1)
\hhline{>{\arrayrulecolor{DarkBackground}}--}
\SetRowColor{LightBackground}
\mymulticolumn{2}{x{5.377cm}}{Full list of allowed tags at \seqsplit{http://en.wikipedia.org/wiki/Help:HTML\_in\_wikitext}}  \tn 
\hhline{>{\arrayrulecolor{DarkBackground}}--}
\end{tabularx}
\par\addvspace{1.3em}

\begin{tabularx}{5.377cm}{x{2.88666 cm} x{2.09034 cm} }
\SetRowColor{DarkBackground}
\mymulticolumn{2}{x{5.377cm}}{\bf\textcolor{white}{Wikipedia Tables}}  \tn
% Row 0
\SetRowColor{LightBackground}
\{| class="name" & Start table \tn 
% Row Count 1 (+ 1)
% Row 1
\SetRowColor{white}
|- & Start row \tn 
% Row Count 2 (+ 1)
% Row 2
\SetRowColor{LightBackground}
! Header 1 & Headings \tn 
% Row Count 3 (+ 1)
% Row 3
\SetRowColor{white}
\mymulticolumn{2}{x{5.377cm}}{! Header 2} \tn 
% Row Count 4 (+ 1)
% Row 4
\SetRowColor{LightBackground}
\mymulticolumn{2}{x{5.377cm}}{|-} \tn 
% Row Count 5 (+ 1)
% Row 5
\SetRowColor{white}
| row 1, cell 1 & Data \tn 
% Row Count 6 (+ 1)
% Row 6
\SetRowColor{LightBackground}
\mymulticolumn{2}{x{5.377cm}}{| row 1, cell 2} \tn 
% Row Count 7 (+ 1)
% Row 7
\SetRowColor{white}
\mymulticolumn{2}{x{5.377cm}}{|-} \tn 
% Row Count 8 (+ 1)
% Row 8
\SetRowColor{LightBackground}
\mymulticolumn{2}{x{5.377cm}}{| row 2, cell 1} \tn 
% Row Count 9 (+ 1)
% Row 9
\SetRowColor{white}
\mymulticolumn{2}{x{5.377cm}}{| row 2, cell 2} \tn 
% Row Count 10 (+ 1)
% Row 10
\SetRowColor{LightBackground}
|\} & End Table \tn 
% Row Count 11 (+ 1)
\hhline{>{\arrayrulecolor{DarkBackground}}--}
\end{tabularx}
\par\addvspace{1.3em}

\begin{tabularx}{5.377cm}{X}
\SetRowColor{DarkBackground}
\mymulticolumn{1}{x{5.377cm}}{\bf\textcolor{white}{Wikipedia Images}}  \tn
% Row 0
\SetRowColor{LightBackground}
\mymulticolumn{1}{x{5.377cm}}{{[}{[}File:img.jpg|thumb|alt=Text|Caption{]}{]}} \tn 
\mymulticolumn{1}{x{5.377cm}}{\hspace*{6 px}\rule{2px}{6px}\hspace*{6 px}Place an image} \tn 
% Row Count 2 (+ 2)
% Row 1
\SetRowColor{white}
\mymulticolumn{1}{x{5.377cm}}{{[}{[}File:img.jpg|thumb|left{]}{]}} \tn 
\mymulticolumn{1}{x{5.377cm}}{\hspace*{6 px}\rule{2px}{6px}\hspace*{6 px}Image on the left} \tn 
% Row Count 4 (+ 2)
% Row 2
\SetRowColor{LightBackground}
\mymulticolumn{1}{x{5.377cm}}{{[}{[}File:img.jpg|thumb|none{]}{]}} \tn 
\mymulticolumn{1}{x{5.377cm}}{\hspace*{6 px}\rule{2px}{6px}\hspace*{6 px}Image with no text flowing around it} \tn 
% Row Count 6 (+ 2)
\hhline{>{\arrayrulecolor{DarkBackground}}-}
\end{tabularx}
\par\addvspace{1.3em}

\begin{tabularx}{5.377cm}{X}
\SetRowColor{DarkBackground}
\mymulticolumn{1}{x{5.377cm}}{\bf\textcolor{white}{Wikipedia Citations}}  \tn
% Row 0
\SetRowColor{LightBackground}
\mymulticolumn{1}{x{5.377cm}}{\textless{}ref\textgreater{}Smith, John. 'Reference Title''. Journal, 2011, p. 1.\textless{}/ref\textgreater{}} \tn 
\mymulticolumn{1}{x{5.377cm}}{\hspace*{6 px}\rule{2px}{6px}\hspace*{6 px}Inline citation.} \tn 
% Row Count 3 (+ 3)
% Row 1
\SetRowColor{white}
\mymulticolumn{1}{x{5.377cm}}{\textless{}ref name="refname"\textgreater{}text\textless{}/ref\textgreater{}} \tn 
\mymulticolumn{1}{x{5.377cm}}{\hspace*{6 px}\rule{2px}{6px}\hspace*{6 px}Named inline citation.} \tn 
% Row Count 5 (+ 2)
% Row 2
\SetRowColor{LightBackground}
\mymulticolumn{1}{x{5.377cm}}{\textless{}ref name="refname" /\textgreater{}} \tn 
\mymulticolumn{1}{x{5.377cm}}{\hspace*{6 px}\rule{2px}{6px}\hspace*{6 px}Use earlier named citation.} \tn 
% Row Count 7 (+ 2)
% Row 3
\SetRowColor{white}
\mymulticolumn{1}{x{5.377cm}}{\{\{citation needed\}\}} \tn 
\mymulticolumn{1}{x{5.377cm}}{\hspace*{6 px}\rule{2px}{6px}\hspace*{6 px}Citation needed.} \tn 
% Row Count 9 (+ 2)
% Row 4
\SetRowColor{LightBackground}
\mymulticolumn{1}{x{5.377cm}}{\{\{reflist\}\}} \tn 
\mymulticolumn{1}{x{5.377cm}}{\hspace*{6 px}\rule{2px}{6px}\hspace*{6 px}Citations list.} \tn 
% Row Count 11 (+ 2)
\hhline{>{\arrayrulecolor{DarkBackground}}-}
\SetRowColor{LightBackground}
\mymulticolumn{1}{x{5.377cm}}{More information at \seqsplit{http://en.wikipedia.org/wiki/Wikipedia:Citing\_sources}}  \tn 
\hhline{>{\arrayrulecolor{DarkBackground}}-}
\end{tabularx}
\par\addvspace{1.3em}

\begin{tabularx}{5.377cm}{X}
\SetRowColor{DarkBackground}
\mymulticolumn{1}{x{5.377cm}}{\bf\textcolor{white}{Wikipedia Useful Pages}}  \tn
% Row 0
\SetRowColor{LightBackground}
\mymulticolumn{1}{x{5.377cm}}{\seqsplit{http://en.wikipedia.org/wiki/Help:Editing}} \tn 
\mymulticolumn{1}{x{5.377cm}}{\hspace*{6 px}\rule{2px}{6px}\hspace*{6 px}Help with editing} \tn 
% Row Count 2 (+ 2)
% Row 1
\SetRowColor{white}
\mymulticolumn{1}{x{5.377cm}}{\seqsplit{http://en.wikipedia.org/wiki/Wikipedia:List\_of\_guidelines}} \tn 
\mymulticolumn{1}{x{5.377cm}}{\hspace*{6 px}\rule{2px}{6px}\hspace*{6 px}Wikipedia guidelines} \tn 
% Row Count 5 (+ 3)
% Row 2
\SetRowColor{LightBackground}
\mymulticolumn{1}{x{5.377cm}}{\seqsplit{http://en.wikipedia.org/wiki/Wikipedia:List\_of\_policies}} \tn 
\mymulticolumn{1}{x{5.377cm}}{\hspace*{6 px}\rule{2px}{6px}\hspace*{6 px}Wikipedia policies} \tn 
% Row Count 8 (+ 3)
% Row 3
\SetRowColor{white}
\mymulticolumn{1}{x{5.377cm}}{\seqsplit{http://en.wikipedia.org/wiki/Wikipedia:Manual\_of\_Style}} \tn 
\mymulticolumn{1}{x{5.377cm}}{\hspace*{6 px}\rule{2px}{6px}\hspace*{6 px}Wikipedia manual of style} \tn 
% Row Count 11 (+ 3)
% Row 4
\SetRowColor{LightBackground}
\mymulticolumn{1}{x{5.377cm}}{\seqsplit{http://en.wikipedia.org/wiki/Wikipedia:Notability}} \tn 
\mymulticolumn{1}{x{5.377cm}}{\hspace*{6 px}\rule{2px}{6px}\hspace*{6 px}Guide to notability} \tn 
% Row Count 13 (+ 2)
% Row 5
\SetRowColor{white}
\mymulticolumn{1}{x{5.377cm}}{\seqsplit{http://en.wikipedia.org/wiki/Wikipedia:Help\_desk}} \tn 
\mymulticolumn{1}{x{5.377cm}}{\hspace*{6 px}\rule{2px}{6px}\hspace*{6 px}Wikipedia help desk} \tn 
% Row Count 15 (+ 2)
% Row 6
\SetRowColor{LightBackground}
\mymulticolumn{1}{x{5.377cm}}{\seqsplit{http://en.wikipedia.org/wiki/User:A.\_B./Useful\_Wikipedia\_links}} \tn 
\mymulticolumn{1}{x{5.377cm}}{\hspace*{6 px}\rule{2px}{6px}\hspace*{6 px}Comprehensive list of useful links} \tn 
% Row Count 18 (+ 3)
\hhline{>{\arrayrulecolor{DarkBackground}}-}
\end{tabularx}
\par\addvspace{1.3em}

\columnbreak
\begin{tabularx}{5.377cm}{x{1.14471 cm} x{3.83229 cm} }
\SetRowColor{DarkBackground}
\mymulticolumn{2}{x{5.377cm}}{\bf\textcolor{white}{Wikipedia Abbreviations and Shortcuts}}  \tn
% Row 0
\SetRowColor{LightBackground}
WP:3RR & Three Revert Rule \tn 
% Row Count 1 (+ 1)
% Row 1
\SetRowColor{white}
WP:AFC & Articles For Creation \tn 
% Row Count 2 (+ 1)
% Row 2
\SetRowColor{LightBackground}
WP:AFD & Articles For Deletion \tn 
% Row Count 3 (+ 1)
% Row 3
\SetRowColor{white}
WP:AGF & Assume Good Faith \tn 
% Row Count 4 (+ 1)
% Row 4
\SetRowColor{LightBackground}
WP:AN & Administrator's Noticeboard \tn 
% Row Count 5 (+ 1)
% Row 5
\SetRowColor{white}
WP:AP & Arbitration Policy \tn 
% Row Count 6 (+ 1)
% Row 6
\SetRowColor{LightBackground}
WP:BOT & Bots \tn 
% Row Count 7 (+ 1)
% Row 7
\SetRowColor{white}
WP:BCRAT & Bureaucrats \tn 
% Row Count 8 (+ 1)
% Row 8
\SetRowColor{LightBackground}
WP:CIVIL & Civility \tn 
% Row Count 9 (+ 1)
% Row 9
\SetRowColor{white}
WP:D & Disambiguation \tn 
% Row Count 10 (+ 1)
% Row 10
\SetRowColor{LightBackground}
WP:DEL & Deletion Policy \tn 
% Row Count 11 (+ 1)
% Row 11
\SetRowColor{white}
WP:NEW & New user log \tn 
% Row Count 12 (+ 1)
% Row 12
\SetRowColor{LightBackground}
WP:NOR & No Original Research \tn 
% Row Count 13 (+ 1)
% Row 13
\SetRowColor{white}
WP:NOT & What Wikipedia is not \tn 
% Row Count 14 (+ 1)
% Row 14
\SetRowColor{LightBackground}
WP:NPA & No Personal Attacks \tn 
% Row Count 15 (+ 1)
% Row 15
\SetRowColor{white}
WP:NPOV & Neutral Point Of View \tn 
% Row Count 16 (+ 1)
% Row 16
\SetRowColor{LightBackground}
WP:PD & Public Domain \tn 
% Row Count 17 (+ 1)
% Row 17
\SetRowColor{white}
WP:PN & Patent Nonsense \tn 
% Row Count 18 (+ 1)
% Row 18
\SetRowColor{LightBackground}
WP:PROD & Proposed Deletion \tn 
% Row Count 19 (+ 1)
% Row 19
\SetRowColor{white}
WP:SELF & Avoid self-references \tn 
% Row Count 20 (+ 1)
% Row 20
\SetRowColor{LightBackground}
WP:V & Verifiability \tn 
% Row Count 21 (+ 1)
% Row 21
\SetRowColor{white}
WP:VAND & Vandalism \tn 
% Row Count 22 (+ 1)
% Row 22
\SetRowColor{LightBackground}
WP:VIE & Voting Is Evil \tn 
% Row Count 23 (+ 1)
\hhline{>{\arrayrulecolor{DarkBackground}}--}
\SetRowColor{LightBackground}
\mymulticolumn{2}{x{5.377cm}}{More at \seqsplit{http://en.wikipedia.org/wiki/User:Kjkolb/Abbreviations} and \seqsplit{http://en.wikipedia.org/wiki/Wikipedia:Shortcuts}}  \tn 
\hhline{>{\arrayrulecolor{DarkBackground}}--}
\end{tabularx}
\par\addvspace{1.3em}


% That's all folks
\end{multicols*}

\end{document}
